\chapter{Distributed Programming}
\section{Analisi del problema}
% Analisi del problema, focalizzando in particolare gli aspetti relativi alla concorrenza.
\subsection{Descrizione del problema}
Viene fornito un programma che implementa un semplice gioco del puzzle, pensato per un solo giocatore. Scelta un’immagine di dimensioni WxH, il puzzle è rappresentato da una matrice NxM di tessere, dove le tessere hanno dimensione W/M e H/N. L’immagine all’inizio del gioco ha le tessere disposte casualmente, non nel posto giusto. Il giocatore mediante una semplice GUI può scambiare la posizione di due tessere, selezionandole in sequenza.\newline

\noindent Lo scopo dell'assignment è realizzare una versione distribuita multi-user, che consenta a un insieme dinamico di utenti in rete di partecipare alla risoluzione del puzzle. Quando il puzzle è risolto, la sessione finisce e tutti i partecipanti devono vedere un messaggio corrispondente (esempio: “Risolto!”).\newline

\noindent I vincoli da rispettare sono i seguenti:
\begin{itemize}
    \item La soluzione deve essere completamente decentralizzata, peer-to-peer (non ci possono essere coordinatori o server centrali).
    \item Gli utenti devono giocare in modo asincrono, concorrente (non c'è un modello a turni).
    \item Un giocatore può entrare nel gioco contattando un qualsiasi altro giocatore.
    \item Devono essere gestite possibili failure - a livello di nodo o rete.
    \item Si può presupporre che ci sia un numero massimo di giocatori (pari a N).
\end{itemize}

\noindent Si richiede di implementare:
\begin{itemize}
    \item Una versione basata sul paradigma ad attori, in particolare utilizzando il Framework Akka.
    \item Una versione basata su Java RMI.
\end{itemize}
