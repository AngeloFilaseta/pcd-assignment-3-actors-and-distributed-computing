\section{Analisi del problema}
% Analisi del problema, focalizzando in particolare gli aspetti relativi alla
% concorrenza.
\subsection{Descrizione del problema}
La specifica del problema richiede la realizzazione dello stesso sistema descritto nell'assignment \#01 - parte 2, sfruttando una soluzione basata su attori: \newline

\noindent Lo scopo dell'assignment è realizzare un programma concorrente che, data una directory \textit{D} presente sul file system locale contenente un insieme di documenti in PDF, provveda a determinare e visualizzare in standard output le \textit{N} parole più frequenti presenti nell’insieme dei documenti, con la rispettiva frequenza, e il numero totale di parole elaborate.\newline
Vanno ignorate (escluse dal conteggio) tutte le parole elencate in un file \textit{F} di testo che viene indicato inizialmente, nel quale è presente una parola da ignorare per riga.\newline
Si presuppone che \textit{D}, \textit{N} e \textit{F} siano specificati per mezzo di una GUI, la quale fornisce all'utente un'interfaccia che permette anche la presentazione dei dati.
